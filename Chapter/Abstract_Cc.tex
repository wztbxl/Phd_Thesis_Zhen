

\setcounter{section}{0}
%==========================================



\chapter*{摘要}

实验研究显示,在高能重离子碰撞中产生了高温高密强耦合的新物质形态——夸克胶子等离子体(Quak-Gluon Plasma, QGP)。在美国布鲁克海文国家实验室建造的相对论重离子对撞机(RHIC)是主要用来研究夸克胶子等离子体的性质及量子色动力学相图的实验装置。在非对心碰撞中,夸克胶子团在旁观质子的运动而产生的极强磁场的作用下,在磁场方向上会出现电荷分离的效应,这就是所谓的手征磁效应(Chiral Magnetic Effect, CME)。实验上致力于寻找重离子碰撞中的手征磁效应已经十多年,但目前为止还没有明确的结论证实它的存在。其中最主要的难点在于手征磁效应研究中很难有效的扣除背景,特别是与共振态的椭圆流相关的背景,它们也会在垂直于磁场的方向上产生类似的分离效应。

最近电荷平衡函数法(Signed Balance Function)被认为可以作为寻找CME的探针,它是基于检验由磁场作用而导致的粒子对横动量差的净取向在垂直、平行反应平面两个方向上起伏的区别而提出来的新方法。在这个方法中提出了一对观测量:$r_{\mathrm{rest}}$ 和$R_{\mathrm{B}}$,这两者对信号、背景的反应不一样。即$r_{\mathrm{rest}}$ 在包含共振态的椭圆流等背景等情况下会增大,而此时
$R_{\mathrm{B}}$会被压低。如果这两个观测量都大于1,那么这种情况就很有可能存在CME。本文利用Toy模型、多相输运模型模型(AMPT)和
异常粘滞流体动力学模型(EBE-AVFD)对该方法的有效性进行了进一步的检验。结果表明该方法通过对$r_{\mathrm{rest}}$ 和$R_{\mathrm{B}}$的结果对比,可以很好的鉴别信号与背景。接下来本文对实验上用于寻找CME的三个基本的方法进行了研究,通过比较它们的核心组成部分发现是相互关联的,它们的核心组成部分在模型中的结果也表明它们对信号和背景的反应是相似的,在此基础上我们认为这三种方法并没有明显的区别。
本文还利用电荷平衡函数法对STAR实验中所采集的Au+Au200GeV 的数据进行研究,$r_{\mathrm{rest}}$,  $r_{\mathrm{lab}}$ 和 $R_{\mathrm{B}}$ 在所有中心度中都大于1,且大于EBE-AVFD模型中没有手征磁效应信号时的结果。这一结果很难被解释为只有背景存在的情况下导致的。

为了能够更好的扣除背景的贡献, 在RHIC的螺线管径迹探测器(The Solenoidal Tracker At RHIC,STAR)实验组进行了同质异荷数对撞实验(Isobar collision),目前正在进行数据分析。这两个Isobar系统分别是:钌($^{96}_{44}$Ru + $^{96}_{44}$Ru)和锆($^{96}_{40}$Zr + $^{96}_{40}$Zr),因为它们具有相同数量的核子数但具有不同的质子数,这就使得两对撞系统会有相同的背景(椭圆流)而具有不同的磁场大小,即两个碰撞系统的CME信号的大小不同。两个Isobar系统通过控制背景不变而改变信号的大小这样就得到了一个研究CME理想的环境。STAR合作组采用了盲分析法以消除由于人为因素而导致的偏差,目前所有用于分析实验数据的程序都已经冻结。因为Isobar的实验数据将会采用多种不同的方法进行分析,因而迫切的需要对不同方法之间的联系与区别以及不同方法对信号的灵敏度进行研究。
由于不同方法之间的优劣一直争论补休,本文利用盲分析中冻结的分析$\gamma$ 关联方法 和 R关联方法的程序,以及电荷平衡函数法在Quark Matter 2019会议报告中所用的相同的方法对EBE-AVFD模型产生的Isobar数据进行分析。研究结果表明三种方法的观测量对于两个系统都会随着CME信号的增大而增大。然而当比较两个系统的比值时,$\Delta \gamma$ 和$R_{lab}$表现出很好的、彼此相当的灵敏度,而$R_{\psi 2}$ and \rb 随着信号的增大却没有表现出明显的变化。关于观测量灵敏度的研究将对STAR实验Isobar数据结果的分析提供很好的参考。

\setcounter{figure}{0}
\setcounter{table}{0}
\setcounter{equation}{0}

\mbox{}\vskip 0.4cm \noindent {\textbf{关键词} \hskip 0.2cm
手征磁效应, 各向异性流, 事件平面, 重离子碰撞, 夸克胶子等离子体, Isobar 对撞

%\clearpage
%\chapter*{\vskip -2cm \centerline {\noindent\LARGE  \textbf{} \hskip 3.0cm \textbf{}}}
%\mbox{}\vskip 0.2cm \centerline {\noindent\LARGE  \textbf{ժ} \hskip 3.0cm \textbf{Ҫ}} \vskip 0.5cm


%\iffalse

%\fi
