\chapter*{\vskip -2cm {\LARGE\centerline{\bf{Abstract}}} }
%\clearpage
%\phantomsection
\addcontentsline{toc}{chapter}{Abstract}

In high energy Heavy-Ion collisions, a new state of hot and dense matter has been created. A series of evidence indicate that such a new state of matter is a strongly coupled Quak-Gluon Plasma (QGP). The Relativistic Heavy Ion Collider(RHIC), built at Brookhaven National Laboratory, is a dedicated machine to study the probperties of QPG as well as QCD phase dragram. 
In non-central collisions, when such domains interplay with the ultra-strong magnetic fields produced by spectator protons, they can induce an electric charge separation parallel in the magnetic field direction — the chiral magnetic effect (CME).
Experimental searches for CME in heavy-ion collisions have been going on for a decade, and so far there is no conclusive evidence for its existence. The major challenge in CME searches is that backgrounds, in particular those related to elliptic flow of resonances, can produce similar enhancement in fluctuation in the direction perpendicular to the reaction plane. 

Recently, the Signed Balance Function (SBF), based on the idea of examining the momentum ordering of charged pairs along the in- and out-of-plane directions, has been proposed as a probe of CME.  
In this approach, a pair of observables is invoked, the two observables give opposite responses to the CME-driven charge separation compared to the background correlations arising from resonance flow and global spin alignment.  Both  $r_{\mathrm{rest}}$ and $R_{\mathrm{B}}$ being larger than unity can be regarded as a case in favor of the existence of CME.
 In this thesis, we will review and provide some updates of  the results from Toy model or other more realistic model,  to confirm whether this method work in more realistic model. 
Since the isobaric collisions will be examined with multiple observables, it is desirable to learn the connection and the difference between them, as well as their  sensitivities to the CME signals. We perform an apple-to-apple comparisons between method kernels.  The comparisons of method kernels in both the toy model and the EBE-AVFD model support the idea that the three observables are equivalent to each other, with their very similar responses to the CME signal as well as the backgrounds.
Up to this point, we see no  obvious advantage of one over the others. 
 Then we implemented this method in Au+Au collisions at 200 GeV at STAR. The results shows that $r_{\mathrm{rest}}$,  $r_{\mathrm{lab}}$ and $R_{\mathrm{B}}$ are larger than unity in all  centralities, and larger than  Event-By-Event Anomalous Viscous Fluid Dynamics (EBE-AVFD) model calculations with no CME implemented.  These findings are difficult to be explained by a background-only scenario. 

To have the background under control, the STAR experiment at RHIC has collected collisions from isobaric collisions and the data analysis is on-going. The two isobaric systems, namely $^{96}_{44}$Ru + $^{96}_{44}$Ru and $^{96}_{40}$Zr + $^{96}_{40}$Zr, have the same number of nucleons and hence similar amounts of elliptic flow, but different numbers of protons, which causes a difference in the magnetic field strength and in turn, a difference in the CME signal. By keeping the background unchanged and varying the signal level, the two isobaric systems provide an ideal test ground for the CME study.
The STAR Collaboration has implemented a blind-analysis recipe to eliminate any unintentional bias in data analyses, and currently all the analysis codes have been frozen as part of the blinding procedure.
For the sensitivity test of the final observables being used in the STAR  blind analyses, we apply the STAR frozen codes for $\gamma$ correlator and $R$-correlator to EBE-AVFD Isobar events with various CME inputs. Beside, the Signed Balance Function method is also studied, this approach  does not participate in the STAR blind analysis, but follows the same procedure as used in the Quark Matter 2019 Conference proceedings.
The results shows that all three methods are sensitive to CME signal for each individual isobar species. However, when studied as the ratio of two isobar, $\Delta \gamma$ and $R_{lab}$ show compatible and decent sensitivity, while $R_{\psi 2}$ and \rb   shows flat response to signal increase.
The sensitivity study can serve as a reference point when interpreting results from STAR's isobaric collisions. 



\iffalse

\fi


\mbox{}\vskip 0.4cm \noindent {\textbf{Key Words} \hskip 0.2cm
chiral magnetic effect, anisotropic flow, event plane , heavy-ion collisions, quark-gluon plasma, Isobar collisions
\cleardoublepage
