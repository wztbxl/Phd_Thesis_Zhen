

\setcounter{section}{0}
%==========================================

\setcounter{figure}{0}
\setcounter{table}{0}
\setcounter{equation}{0}

\chapter{总结与展望}
% To add a non numbered chapter
%\addcontentsline{toc}{第一章}{相对论重离子碰撞}
% To insert this section on the table of contents

在本文当中给出了STAR实验中\sNN = 54.4 GeV 金-金对撞当中的双电子谱测量结果。主要物理目标集中在对夸克胶子等离子体温度的测量。在中等质量区间,夸克胶子等离子体热辐射产生的双电子是双电子额外产额谱的主要来源,通过对其拟合的方式可以抽取介质的温度。在不同的中心度下中等质量区间当中抽取的到的温度均在320 MeV左右,高于格点量子色动力学中理论计算得到相变温度。这是在RHIC上首次通过测量双轻子谱的方式对夸克胶子等离子体的温度进行直接测量。

在 $\rho$介子衰变占主要产额的低质量区间,通过联合$\rho$介子质量方程和热辐射产额描述方程进行拟合的方式对此区间的介质温度进行了抽取。结果发现在多个不同的对撞系统和对撞能量下,此区间内的温度接近于格点量子色动力学中理论计算得到相变温度,这意味着$\rho$介子产生于从夸克胶子等离子体相向强子物质相转变的过程中,这也是在实验上首次有证据表明$\rho$介子产生于这个过程当中。这个区间当中的双电子额外产额谱也和理论模型的计算结果进行了比较。

同时进行的\sNN = 27 GeV 金-金对撞当中的双电子谱测量也给出了在相同区间内的温度测量结果。在这两个不同的对撞能量下所测量得到的温度在低质量区间和中等质量区间中彼此符合的很好且都为观测到明显的温度随着中心的的变化。当对撞的能量不同时,其所具有的重子化学势也不相同。STAR的束流能量扫描第二阶段已经完成取数,有着高统计量的更低能量下的对撞(\sNN = 3.3-19.6 GeV)以及固定靶(\sNN = 3-13.7 GeV)实验数据已经采集完成,双电子的分析已经在进行当中。这些结果可以让我们对温度和重子化学势的依赖关系有一个系统的研究。

除了像本文中的传统的在整个横动量区间的双轻子谱测量,双轻子测量在不同的中心度以及横动量区间有着很多不同的物理目标。例如在偏心和极偏心对撞(60-80\%以及80-100\%)的低横动量区间当中可以对Breit-Wheeler 过程产生的双电子谱进行测量。STAR前期的实验结果已经观测到了此物理过程\cite{STAR:2019wlg, STAR:2018ldd},而在\sNN = 54.4 和 200 GeV当中的测量也正在进行当中,初步结果已在Quark Matter 2022中给出报告。同样在STAR束流能量扫描第二阶段的数据分析当中也可以对此区间进行双电子谱的测量,研究相关物理量与束流能量的依赖关系。

在物理分析之外,本文也给出了STAR前向探测器升级当中sTGC探测器在布鲁克海文国家实验室的测试结果以及相关模拟以及重建的软件包编写工作的进展。其中在布鲁克海文国家实验室的模型机测试结果最终确定了运行过程中的工作气体。其效率测试结果也达到了预期的设计要求。同时基于探测器的设计,重建粒子集中位置的软件包sTGC cluster finder框架已经编写完成,正在利用2022年所采集的数据对其进行测试。

STAR的前向升级拓展了STAR实验在前向的接收度。和之前相比一个很重要的升级便是在STAR的前向部分有了径迹探测的能力。这使得一些新的物理测量得以成为可能。在STAR上进行过对gluon saturation的测量\cite{STAR:2021fgw}。因为在其分析所用的数据采集过程当中,STAR前向仅有前向介子谱仪可以用来进行物理测量。所以只通过双 $\pi^0$关联的方式进行了测量,并且因为前向介子谱仪的触发本身有着 $\rm{p_T > 1~ GeV/c}$的要求,这限制了所能测量到的x区间。当STAR前向探测器升级完成之后便可以通过双强子关联以及光子-喷注或者光子-强子关联的方式进行测量。有效地拓展了STAR在前向区间内的物理探测手段。而且在此测量当中,一个很大的系统误差来源便是通过前向介子谱仪重建$\pi^0$时所带来的系统误差, STAR探测器的前向升级可以有效地减少这个来源的系统误差。在STAR前向物理升级的帮助下在前向快度区间可以有着更多更新的物理测量,这也对以后电子-离子对撞机(Electron-Ion Collider, EIC)的升级以及物理测量有着指导意义。

