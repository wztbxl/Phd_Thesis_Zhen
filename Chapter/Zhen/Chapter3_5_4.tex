\subsection{Drell-Yan 过程}
当对撞的质心能量降低的时候,Drell-Yan过程产生的双电子产额和由璨夸克产生的双电子产额处在同一个数量级。所以在模拟过程中不能忽略来自于Drell-Yan过程的双电子。在进行归一化时所用到的归一化公式和璨夸克模拟的模拟时所用到的类似。Drell-Yan过程的模拟也面临着和璨夸克模拟时类似的问题,缺少截面$\sigma_{DY}$的测量。

在STAR之前的双电子谱测量当中,\sNN = 19.6 GeV的测量能量与本分析接近且在其强子衰变模拟中包含了Drell-Yan过程,其中$\sigma_{DY}$为 9.88 nb\cite{STAR:2015zal}。而在Pythia当中,默认的\sNN = 19.6 GeV下的$\sigma_{DY}$为13.44 nb。这两个值之间的比值被用作修正因子来修正\sNN = 54.4 GeV Pythia默认的$\sigma_{DY}$的值来外推得到\sNN = 54.4 GeV中的$\sigma_{DY}$。外推公式和结果如式\ref{eq:DY}所示。
\begin{equation}
    \label{eq:DY}
    \begin{split}
        \rm{ \sigma_{DY} = \sigma_{DY}^{Pythia}*\frac{\sigma_{DY}^{19.6~GeV~papaer}}{\sigma_{DY}^{19.6~GeV~Pythia}} = 19.25 nb } \\
        \rm{  \sigma_{DY}^{54.4~GeV~Pythia} = 26.19 nb }
    \end{split}
\end{equation}