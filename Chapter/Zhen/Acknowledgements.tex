\chapter{致谢}

时光荏苒,转眼间五年多的研究生生涯即将走到尽头,刚入学时师兄师姐感叹自己年轻和辛苦带自己的经历仿佛就在昨天,结果一转眼自己也成了感叹新生年轻的那帮老狗。回首往事,不能说是兢兢业业,只能说是摸鱼成性、忝列门墙。提笔至此,这些年的一幕幕像走马灯一样在脑海中闪过,有欢笑有泪水,有幸福的美好时光也有至暗的崩溃时刻。但幸运的是在这些年里遇到了许多可爱的人,在你们的帮助下我才能健康正常的走过这段岁月。

首先要感谢我的导师许长补研究员和杨驰教授,是你们带我走进了重离子对撞物理这个奇妙和美丽的世界,让我开始对其有了认识并且在你们的指导下开始研究学习,也让我有机会可以圣地巡礼。在工作中这些年来你们认真严谨的治学态度,在科学上敏锐的思维和洞察力以及勤恳踏实的工作态度让我获益良多。在生活上你们的关照也帮助着我度过了这几年中的难关。学高为师,身正为范是对你们最真实的写照。

同样诚挚的感谢布鲁克海文国家实验室的阮丽娟研究员,James Daniel Brandenburg博士。在BNL的时光中二位的指导和帮助是鞭策我前进的动力。阮姐您开阔的科学视野、认真勤勉的工作态度是我前进路上的榜样,您的言传身教和勤勉认真的传说是我鞭策自己更加努力的动力。感谢Daniel在我如此不靠谱的情况下还细致的指导我在分析和探测器方面的工作,感谢你的帮助。

感谢徐庆华老师,研一的时候最开始是您带我走进了高能物理的大门,让我有机会一窥内里的奇妙。感谢大师兄梅金成,研一的时候是你不厌其烦的教我ROOT、STAR分析入门让我能尽快的适应科研的学习生活,同时作为朋友和你的相处也让我受益颇丰。

感谢山大的各位老师们的关怀和帮助,你们的科研态度和营造的科研氛围对我影响深远。感谢山大能在我本科四年烂完了以后有从头来过的机会。

除了尊敬的师长,朋友们的陪伴也是这些年来能让自己一直走下去的支撑。人永远不会是一个孤岛,是你们让我这个幼小脆弱的孩子慢慢地离开孤岛,走入人和人之间的联系。
特别感谢王皓月、聂熙伦老师、山大心理咨询中心的老师们、张慧凝、艾欣、李玉莹、秦天真的帮助和张勇、季祥、刘美麟、陈淑霞、包汉卿、伊江山、孙蕊、孙超越二十年来的陪伴,没有你们的帮助可能活到毕业这句话就不仅仅是聊天时引人一哂的玩笑。感谢在BNL遇到的可爱的小伙伴们:褚晓璇、王栋、常婉、李洋、胡昱、王鹏飞、林裕富、王添翼、叶早晨、张正桥、常子龙、涂周顿明、高翔、陈玎、金小海、申迪宇、黄德荃、郗宝山。第一次离开山东就是去往异国他乡的BNL,是你们让我尽快地适应了BNL的生活,让在BNL的生活充满了欢乐。尤其感谢褚晓璇和王栋,在BNL的时候是你们拉了我最关键的一把。感谢Isaac Upsal,Prashanth Shanmuganathan、李洋在sTGC工作上给予我的帮助。感谢叶早晨在我分析上给予的帮助。感谢BNL和STAR合作组良好的科研环境让我受益颇丰,感谢所有在美国的时候帮助过我的人们。

感谢在山大认识的人们,有你们的陪伴让研究生的旅途不再单调。感谢山大STAR组的各位:梅金成、沈付旺、聂茂武、杨钱、王帅、孔凡刚、朱展文、陈佳、苟兴瑞、李长丰、许一可、纪赵惠子、闫高国、于毅、孙川、何洋、张晴、高涛亚、张梦雪、张宜新、汪杰克,是你们组成STAR组和谐友爱的大家庭。感谢师出同门师弟师妹们:史迎迎、王晓凤、王永红、沈丹丹、包贤文,带新手的日子总归是欢乐的,你们的优秀也让我感到骄傲,愿君共勉。感谢虽不是同一个组但在给自己生活添加了几分色彩的小伙伴们:姜候兵、周航、韩婷婷、袁睿、黄文昊、郑杜鑫、季昊、韩靖宜、张晓、张子睿、熊秉诚、姜彪、高铭升、时倩倩、翟云聪、姚志鹏、曾凡蕊、于明玉。

感谢Nihgtwish,Pink Floyd。是你们的音乐让我在毕业论文写作期间不断地提振精神,完成毕业论文的写作。

感谢我的家人们,是你们这些年来的无私的奉献和支持让我走到了现在。感谢你们对我性格上的缺陷的包容和理解,希望你们在以后的日子里可以健康幸福。往事不可追,以前给你们带来的困扰希望就留在昨天,以后的日子里自己不再会给你们带来困扰。希望爷爷您也能看到这篇文章,您走的时候问我有没有考上博士生,现在我已然接近毕业,希望您知道以后能有所宽慰。

王木一,和你相处已有十年有余,但十多年中将你放在泥潭不闻不问,而你也一步步将我拉向泥潭。十余年前你我放弃了成长,十余年后不得不重新面对被我们埋葬而避而不见的问题。衷心的祝愿我们能友好地相处,不再互相厌恶。感谢十多年来你的陪伴,虽然世界上最不想见到的人就是你,但是最后把自己从泥潭里拉出来还是需要你的力量。希望以后的日子里我们可以像Madeline和Badeline一样达成和解,一起攀上横亘在面前的Mount Celeste。

似有千言却不知如何落笔,或许生活就是这样,总有些事情不能完美,不论自己是什么样子的接受自己才是最重要的。

GL HF
\\
\\
\rightline{山高路远,江湖再见}
\rightline{王桢}