\section{系统误差分析}

在本分析当中,系统误差主要有以下几种来源:
\begin{itemize}
    \item 1.背景分布的估计
    \item 2.强子污染
    \item 3.效率修正
    \item 4.强子衰变模拟
\end{itemize}

在本分析中,同号分布经过电子对接收度修正后被用作背景分布,电子对接收度修正因子由混合事例的方法得到,详见\ref{chap:mixedEvent}。由电子对接收度修正因子带来的系统误差通过改变混合事例方法的分库方法来进行估计。默认情况下数据根据不同的$V_z$和事例平面被分为20 $\times$ 12组,每组350个事例。改变$V_z$和事例平面的分组数和每组的事例数之后再进行混合,得到的新的电子对接收度因子再对同号分布进行修正并且进行背景扣除,从而得到双电子信号分布。由新的电子对接受度因子修正的到的双电子谱和用默认的接受度修正因子得到的双电子谱之间的差距被用作背景分布估计的系统误差。

当做电子鉴别的时候会有一些强子仍然通过电子鉴别的判选条件在分析中被用作电子进行配对,如果这些强子是和其他粒子有着某种关联性的话,他们和其他强子的配对也有可能贡献到最终的信号当中。为了估计强子污染在最终信号当中的影响,首先需要得到在不同的动量区间电子的纯度,关于电子纯度已经在当中被讨论过。同时也需要产生一个纯强子的数据样本。这个由纯强子组成的数据样本和电子样本一起经过本分析重建电子对的流程,得到的分布用来估计在最后的信号当中电子-强子对和强子-强子对的影响。在STAR以前 \sNN = 200 GeV金-金对撞的双轻子分析当中,在1-3${\rm GeV/c^2}$的区间,强子污染所占的影响$<5\%$。在本分析中也使用此数据来估计强子污染对\sNN = 200 GeV数据的影响。

效率修正主要包括单电子径迹探测效率和双电子重建效率两部分组成。其中单电子径迹探测效率主要包括时间投影室的探测效率、飞行时间探测器匹配的效率和电子鉴别的效率。时间投影室的探测效率主要对一条径迹的以下三个量进行判选得到,nHitsFit,nHitsDedx,dca。在计算系统误差的时候,本分析通过调整在计算效率的时候这些量的判选条件来估计这三个量对时间投影室径迹探测效率带来的影响。对于飞行时间探测器的匹配效率,通过调整在选择纯电子数据样本时候对于双电子对质量的判选条件来对其系统误差进行估计。电子鉴别效率已经在\ref{chap:efficiency}中进行过讨论。对于\nSigmaE 判选条件,在进行双电子效率修正时使用的是拟合曲线作为输入,所以拟合曲线的一倍$\sigma$区间就被用来估计系统误差。对于$1/\beta$的判选条件,如图\ref{fig:beta_cut_eff_080}所示,bin counting和拟合两种不同方法计算得到的效率之间的差别被作为系统误差。
