

\setcounter{section}{0}
%==========================================

%\section*{第一章:相对论重离子碰撞}


\chapter{相对论重离子碰撞}
% To add a non numbered chapter
%\addcontentsline{toc}{第一章}{相对论重离子碰撞}
% To insert this section on the table of contents

\setcounter{figure}{0}
\setcounter{table}{0}
\setcounter{equation}{0}

\chapter*{Abstract}
% To add a non numbered chapter
\addcontentsline{toc}{chapter}{Abstract}
% To insert this section on the table of contents



Strong interaction forces (also known as nuclear forces) are one of the four fundamental interaction forces in nature that bind nucleus (protons and neutrons) to form atomic nuclei and dominate more than 90\% of the visible matter in nature. 
Quantum Chromodynamics (Quantum Chromodynamics, QCD) is a modern theory that describes strong interaction forces. The basic unit of the substance, quarks and glues, is confined to the nucleus by strong interaction forces, so no free quarks and gluons are found in nature. The phase diagram of high temperature and high-density nuclear material is the frontier and hot spot in the field of nuclear physics research.
Lattice QCD predicts that at high temperature and low baryon chemical potential, the phase transition between the hadron matter and the quark gluon plasma occurs  is smooth crossover, while the model predicts that at the high baryon chemical potential, the phase transition between them is a first-order phase transition.  Therefore, if the first-order phase transition does exist, then there must be an end point in the end of the first-order phase transition line to the smooth crossover, which is called the QCD critical point. 

The experimental confirmation of QCD critical point will be a milestone in the exploration of the phase structure of strong interaction substances, which is of great scientific significance. In order to take a leading position in this potentially significant discovery research direction and make a breakthrough, various countries have built large particle detectors and carried out heavy ion collision experiments (including: RHIC-STAR beam energy scanning experiment in the United States, CBM experiment in Germany, NICA experiment in Russia, J-PARC experiment in Japan and CEE experiment of the external target of CSR in Lanzhou, China), the main physical goal is to study the structure of high temperature and high-density nuclear material phase diagram, search for the critical point.

In the first phase of Beam Energy Scan (BES) program, Relativistic Heavy Ion Collider (RHIC) located at Brookhaven National Laboratory (BNL), in the Unites States used the STAR detector to complete data collection of 7.7, 11.5  14.5, 19.6, 27, 39, 54.4, 62.4 and 200 GeV by accelerating heavy ions. This allows us to explore the phase diagram in a broader range.


 
 This thesis is organized as follows. The first chapter mainly introduced the motivation, the experimental measurements and the presentation of  these observables in statistics and probability. In the second chapter, we briefly introduced the structure and function of the STAR detector and its sub-detectors at RHIC. The third chapter mainly introduces the details of experimental analysis, data selection, event selection, particle identification, definition of centrality,  multiplicity distribution of net protons and model introduction. The fourth chapter mainly studies the effect of some effects on the results, such as the centrality bin width correction of and the limited detector efficiency correction. In the last chapter, we will present the calculation results of the experiment, including the centroid model and fixed target mode in the Au+Au collisions and the Cu+Cu collisions, and discuss the development prospects of the experiment.\\


\textbf{Key Words:} Heavy Ion Collisions; QCD Phase Transition; QCD Critical Point; Higher Order Cumulants; Correlation Function