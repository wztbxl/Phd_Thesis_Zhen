\chapter*{致谢}
\setcounter{section}{0}
\setcounter{figure}{0}
\setcounter{table}{0}
\setcounter{equation}{0}

% To add a non numbered chapter
\addcontentsline{toc}{chapter}{致谢}


大学毕业伊始,怀着好奇、憧憬来到了桂子山山头,来追逐我的未知——当时我并不懂什么是科研、什么是博士,只是不想工作、逃避就业。
当老师问我读不读博、有没有意向出国的时候,我开始认认真真的思考自己真正想要的是什么?自己的目标是什么?高中的时候,老师告诉我们大学就轻松了。然后,我的大学生活没有什么压力、没有明确的目标,也没有认认真真的把大学的课程学好,只是玩着、混着。所以当我说自己是学渣的时候也就只有大学同学能理解我这个学渣是不是浪得虚名了。
那么,我的目标是什么呢?我自己又能做什么呢?这个问题一辈子去思考、追寻。
人生30年,硕博七年,遇见了很多人、很多事,让我经历了、学习了、收获了很多,我心中充满了感激。在次博士毕业之际,我相对这些人表示我由衷的感激。


首先感谢我的导师吴元芳教授。她以渊博的学识、严谨的科研态度指导着我、鞭笞着我、鼓励着我在科研的道路上克服一个又一个的困难,培养我独立学习、科研的能力。是她的敦敦教诲让我急躁的心得以平复,让我明白科研需要一个良好的心态、严谨的治学态度,不能急功近利,不能急躁。这些年,吴元芳老师为了学生的科研、学习而劳累奔波,祝愿吴老师身体健康、生活愉快、阖家欢乐。

感谢美国布鲁克海文国家实验室(BNL)的唐爱洪研究员。在BNL的两年多时间里,让我收获颇丰。您在我的科研上、生活上给予了极大的帮助,是您一步步的引导我走向这个科研的殿堂。还记得那时我们一起在您办公室讨论问题的场景,记得在青岛、武汉Quark Matter 报告前在您房间一起熬夜试讲的情形。您对科研的热情、热爱以及为了科研的执着、严格的要求,让我逐渐形成了良好的科研习惯,在您的熏陶下让我也希望能够为科学事业做出一点力贡献。

感谢李治明老师。研究生一开始的时候因为基础比较差,是您手把手的指导我做科研,让我逐渐熟悉了解了科研。感谢罗晓峰老师,您丰富的科研经验和敏锐的科研嗅觉让我在实验数据分析少走了很多弯路。是你们以丰富的实验研究经验,长期耐心的指导我,培养了我数据分析的技巧和能力。是你们在我知所措的时候指引了我前进的方向,指导、督促我要细心、严谨、认真的做好每一件事,引领我在科研之路前行。感谢王亚平老师,还记得研一的时候跟张恒英一起去听您上的蒙塔卡洛的课。


感谢刘峰老师、许明梅老师、施梳苏老师、裴骅老师、王亚平老师、喻宁老师、孙旭师兄,感谢您们组会上给予我的宝贵意见,是你们的督促、指导让我能够做的更好。
感谢刘峰老师、马亚老师在研究生期间给了我勤助的机会。
感谢粒子所的所有老师们为我提供了良好的学习、科研环境,是你们的无私奉献让我们能够一心一意的科研、学习。

感谢柯宏伟师兄。感谢您在研二时发邮件咨询关于实验数据分析的热情、认真的回复,感谢您在BNL期间的不厌其烦的指导以及在生活上的关怀,还记得最开始您为我梳理KFParticles程序包时的情形,希望以后还有机会再聚。
感谢Maksym~Zyzak、Yuri~Fisyak,在BNL的前半年里,在利用KFParticles程序包寻找新粒子时是你们一直在帮助着我,容忍着我这蹩脚的口语,虽然后面放弃了,但学到的东西永远不会过时。以后我会继续做下去,希望能够有所突破。

感谢Jinfeng~Liao, Shuzhe~Shi, Gang~Wang的讨论与指导。感谢您们在CME方法比较工作中给予的帮助。其中三种方法相关性的推到主要由Gang~Wang老师推导,Shuzhe~Shi则对电荷平衡函数相关性的推导做出了主要贡献。
因为今年毕业时间仓促,很遗憾没能协调好时间邀请您们参加我的毕业答辩。希望疫情早点结束,等你们回国了欢迎以后到桂林来旅游。


特别感谢Zhangbu~Xu, Lijuan~Ruan, 唐爱洪老师,谢谢您们在科研上对我的指导,谢谢您们为了我们这些中国留学生的住宿和housing各种协商、争取,甚至还轮流到我们宿舍检查卫生以汇报给housing。谢谢您们,您们辛苦了。

感谢申迪宇,你的到来让BNL多了很多欢乐。你的厨艺是真可以,特别是炒牛肉;当然,牛角椒炒茄子听说也让大家印象深刻啊,可惜我没吃到(偷笑)。也非常感谢你一直以来的帮助。其中QM结果,关于分辨率修正的部分是申迪宇验证的;方法比较部分电荷平衡函数推导是申迪宇推到的,以及文中核心部分的比较也是你计算的,谢谢你的这些帮助。


此外在BNL期间,认识了很多伙伴。
首先感谢舍友杨钱、王鹏飞、金小海,是你们让在Apartment 8A多了很多乐趣。
感谢楚晓璇、常婉,跟你们在8A搭伙的日子还是很开心的,虽然你们总是批评我,那时候一起跑步的日子还是很值得怀念的。
感谢张金龙师兄,感谢您在学习、科研、生活上的指导与帮助,感谢您带我去练车、去买菜、去爬山、去NYCity逛——这似乎是我唯一一次去NYCity了,您就像个大哥哥,为我们的事操心、奔波。
感谢Zaochen~Ye,你总是那么风趣,你的到来让8A多了欢乐,特别是德州扑克的引入,成为了那段时间大家每周最期待的活动。(也导致了我跟金龙连续两周一局没有赢过的记录)。
感谢涂周顿明(空),虽然你比我小,不过还是叫你师兄了。感谢你在科研上的指导,感谢你带我去滑雪、去烧烤、去普林斯顿大学玩,希望以后有机会跟你一起去冲浪。
感谢Zilong~Chang带我去坐飞机,环着长岛一直到NYCity,还看到了自由女神像。由于疫情,这是我在美国玩的最后一个地方了。非常谢谢你能脚上我去做你开的飞机。希望以后还有机会坐的开的飞机看风景。
感谢张正桥,一起去抓螃蟹、钓鱼、赶海的日子真是非常怀念的,可惜我还没有海钓成功过。等以后有机会了再跟你去钓鱼啊,要一次两条那种。
感谢王祯、胡昱、习宝山、李洋、黄德荃、陈玎、张春建,希望以后还能一起吃火锅、喝酒、爬山、散步。
感谢高翔,巨欣跃、纪媛婧,Yicheng Feng,与你们的交流让我收获良多。
感谢在BNL认识的、不认识的人,感谢你们一直以来的陪伴与帮助。

感谢粒子所一起学习、研究、生活的伙伴们。感谢潘雪师姐、陈丽珠师姐、张凡师兄、杨鹏师兄、赵烨印师兄、涂彪师兄,是你们不厌其烦的解答让我在学习之路上少走了很多弯路。感谢张凡师兄、杨鹏师兄在研究生开始时手把手的带我们学程序、熟悉root。最开始接触实验数据的时候什么都不懂,涂彪师兄是我在实验数据分析这一块的领路人,非常感谢他的无私的帮助。
感谢张雁华、张恒英、李笑冰、张东海、吴锦、龙芬、赵佳,有你们的陪伴,让这略显单调的科研生活增添了很多欢声笑语。特别感谢李笑冰、张东海在我出国期间、毕业期间的各种帮忙。感谢这些年一起打过球、爬过山、骑过车的所有伙伴,希望以后我们还有机会一起学习、娱乐。


最后,感谢我的父亲林太英、母亲钟玉琼,是你们含辛茹苦的把我拉扯大;是你们为我提供了一个温暖的港湾,让我可以无后顾之忧的去追逐我的梦。千言万语都不足以述说我心中对您们的感激,只希望从今往后能够好好保重身体,开开心心的生活。希望以后您们不要再为我担心,您们的孩子我已经长大了,以后让我来照顾您们。感谢弟弟林裕旺、弟妹何艳红的理解和支持。
%感谢我的女朋友陈丹丹,你的督促我不会忘记,不颓废不找借口,做一个有理想有抱负的追梦人。
感谢所有的家人们,谢谢你们的支持与鼓励,足够的勇气去迎接未知的挑战。

感恩所有的遇见,人生的旅途因与你们相遇而丰富多彩。愿你幸福安康!

未来未来,只要你有勇气、有魄力、有行动,你的未来将由你自己塑造。
