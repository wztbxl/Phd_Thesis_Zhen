\chapter*{附录 :电荷平衡函数法中$\sigma(\Delta B_y)$ 和$\sigma(\Delta B_x)$ 的推导}
\setcounter{section}{0}
\setcounter{figure}{0}
\setcounter{table}{0}
\setcounter{equation}{0}

% To add a non numbered chapter
\addcontentsline{toc}{chapter}{附录 :电荷平衡函数法中$\sigma(\Delta B_y)$ 和$\sigma(\Delta B_x)$ 的推导}

%\section*{ $\sigma(\Delta B_y)$ 和$\sigma(\Delta B_x)$ 的引入}
\label{appendix1}

为了计算 $\Delta B_y$,  $\sigma(\Delta B_y)$的标准差,首先做了以下展开:
\begin{eqnarray}
(N_{x(\alpha\beta)}-N_{x(\beta\alpha)})^2 &=& 
\Big\{\sum_{\alpha,\beta} {\rm Sign}[\sin(\Delta\phi_\alpha)-\sin(\Delta\phi_\beta)] \Big\}^2  \\
&=& \Big\{\sum_{\alpha,\beta} {\rm Sign}[\sin(\Delta\phi_\alpha)-\sin(\Delta\phi_\beta)] \Big\} \Big\{\sum_{\alpha',\beta'} {\rm Sign}[\sin(\Delta\phi_{\alpha'})-\sin(\Delta\phi_{\beta'})] \Big\} \nonumber \\
&=& \sum_{\alpha\neq\alpha'}\sum_{\beta\neq\beta'} {\rm Sign}[\sin(\Delta\phi_\alpha)-\sin(\Delta\phi_\beta)]\times{\rm Sign}[\sin(\Delta\phi_{\alpha'})-\sin(\Delta\phi_{\beta'})]  \nonumber \\
& &+ \sum_{\alpha}\sum_{\beta\neq\beta'} {\rm Sign}[\sin(\Delta\phi_\alpha)-\sin(\Delta\phi_\beta)]\times{\rm Sign}[\sin(\Delta\phi_{\alpha})-\sin(\Delta\phi_{\beta'})]  \nonumber \\
& &+ \sum_{\alpha\neq\alpha'}\sum_{\beta} {\rm Sign}[\sin(\Delta\phi_\alpha)  \\
&& -\sin(\Delta\phi_\beta)]\times{\rm Sign}[\sin(\Delta\phi_{\alpha'})-\sin(\Delta\phi_{\beta})]  
+ \sum_{\alpha}\sum_{\beta} 1. \label{eq:A2}
\end{eqnarray}
然后对式.~\ref{eq:A2} 中的各项取平均,其计算方式与式.~\ref{eq:integral_result}所用方法类似。即第一项有:
\begin{eqnarray}
& &\Big\langle\sum_{\alpha\neq\alpha'}\sum_{\beta\neq\beta'} {\rm Sign}[\sin(\Delta\phi_\alpha)-\sin(\Delta\phi_\beta)]\times{\rm Sign}[\sin(\Delta\phi_{\alpha'})-\sin(\Delta\phi_{\beta'})]\Big\rangle  \nonumber \\
&=& N_\alpha(N_\alpha-1)N_\beta(N_\beta-1)\times
\Big\{
\int_{-\pi/2}^{\pi/2} 
\Big[\int_{-\pi-\Delta\phi_\alpha}^{\Delta\phi_\alpha} \frac{dN}{d\Delta\phi_\beta}d\Delta\phi_\beta-\int^{\pi-\Delta\phi_\alpha}_{\Delta\phi_\alpha} \frac{dN}{d\Delta\phi_\beta}d\Delta\phi_\beta\Big]\frac{dN}{d\Delta\phi_\alpha}d\Delta\phi_\alpha \nonumber \\
&  & \:\:\:\:\:\:\:\:\:\:\:\:\:\:\:\:\:\:\:\:\:\:\:\:\:\:\:\:\:\:\:\:\:\:\:\:\:\:\:\:\:\:\:\:\:\:\:\:\:\:\:\:\:\: +\int_{\pi/2}^{3\pi/2}
\Big[\int_{\pi-\Delta\phi_\alpha}^{\Delta\phi_\alpha} \frac{dN}{d\Delta\phi_\beta}d\Delta\phi_\beta-\int^{3\pi-\Delta\phi_\alpha}_{\Delta\phi_\alpha} \frac{dN}{d\Delta\phi_\beta}d\Delta\phi_\beta\Big]\frac{dN}{d\Delta\phi_\alpha}d\Delta\phi_\alpha
\Big\}^2 \nonumber \\
&=& N_\alpha(N_\alpha-1)N_\beta(N_\beta-1)\Big[\frac{8}{\pi^2}(1+\frac{2}{3}v_2)(a_{1,\alpha}-a_{1,\beta}) \Big]^2.
\label{eq:A3}
\end{eqnarray}
第二项可以写成:
\begin{eqnarray}
& &\Big\langle\sum_{\alpha}\sum_{\beta\neq\beta'} {\rm Sign}[\sin(\Delta\phi_\alpha)-\sin(\Delta\phi_\beta)]\times{\rm Sign}[\sin(\Delta\phi_{\alpha})-\sin(\Delta\phi_{\beta'})]\Big\rangle  \nonumber \\
&=& N_\alpha N_\beta(N_\beta-1)\times
\Big\{
\int_{-\pi/2}^{\pi/2} 
\Big[\int_{-\pi-\Delta\phi_\alpha}^{\Delta\phi_\alpha} \frac{dN}{d\Delta\phi_\beta}d\Delta\phi_\beta-\int^{\pi-\Delta\phi_\alpha}_{\Delta\phi_\alpha} \frac{dN}{d\Delta\phi_\beta}d\Delta\phi_\beta\Big]^2\frac{dN}{d\Delta\phi_\alpha}d\Delta\phi_\alpha \nonumber \\
&  & \:\:\:\:\:\:\:\:\:\:\:\:\:\:\:\:\:\:\:\:\:\:\:\:\:\:\:\:\:\:\:\:\:\:\:\: +\int_{\pi/2}^{3\pi/2}
\Big[\int_{\pi-\Delta\phi_\alpha}^{\Delta\phi_\alpha} \frac{dN}{d\Delta\phi_\beta}d\Delta\phi_\beta-\int^{3\pi-\Delta\phi_\alpha}_{\Delta\phi_\alpha} \frac{dN}{d\Delta\phi_\beta}d\Delta\phi_\beta\Big]^2\frac{dN}{d\Delta\phi_\alpha}d\Delta\phi_\alpha
\Big\} \nonumber \\
&=& N_\alpha N_\beta(N_\beta-1)\Big[\frac{1}{3}-\frac{8}{\pi^2}(1+v_2)a_{1,\beta}(a_{1,\alpha}-a_{1,\beta}) \Big].
\label{eq:A4}
\end{eqnarray}
类似的第三项为:
\begin{eqnarray}
& &\Big\langle\sum_{\alpha\neq\alpha'}\sum_{\beta} {\rm Sign}[\sin(\Delta\phi_\alpha)-\sin(\Delta\phi_\beta)]\times{\rm Sign}[\sin(\Delta\phi_{\alpha'})-\sin(\Delta\phi_{\beta})]\Big\rangle  \nonumber \\
&=& N_\alpha(N_\alpha-1) N_\beta\Big[\frac{1}{3}+\frac{8}{\pi^2}(1+v_2)a_{1,\alpha}(a_{1,\alpha}-a_{1,\beta}) \Big].
\label{eq:A5}
\end{eqnarray}
最后一项为:
\begin{equation}
\sum_{\alpha}\sum_{\beta} 1 = N_\alpha N_\beta. 
\label{eq:A6}
\end{equation}
又根据式~\ref{eq:by}, 我们可以得到
\begin{equation}
\sigma^2(\Delta B_y) = \frac{M^2}{N_+^2 N_-^2} \langle  (N_{x(+-)}-N_{x(-+)})^2  \rangle.
\label{eq:A7}
\end{equation}
为了间并期间,我们假设 $N_+ = N_- = M/2 \gg 1$ 、 $v_2 \ll 1$, 并把式~\ref{eq:A3}、 \ref{eq:A4}、 \ref{eq:A5} 、 \ref{eq:A6} 放入公式.~\ref{eq:A7}, 可以得到:
\begin{eqnarray}
\sigma^2(\Delta B_y) &\approx& \frac{4(M+1)}{3}+ \frac{64M^2}{\pi^4}\Big[(1+\frac{2}{3}v_2)^2+\frac{\pi^2(1+v_2)}{4M}\Big](a_{1,+}-a_{1,-})^2 \nonumber \\
&\approx& \frac{4M}{3}+ \frac{64M^2}{\pi^4}(1+ \frac{4}{3}v_2)(a_{1,+}-a_{1,-})^2.    
\end{eqnarray}
同理可得:
\begin{eqnarray}
\sigma^2(\Delta B_x) &\approx& \frac{4(M+1)}{3}+ \frac{64M^2}{\pi^4}\Big[(1-\frac{2}{3}v_2)^2+\frac{\pi^2(1-v_2)}{4M}\Big](v_{1,+}-v_{1,-})^2 \nonumber \\
&\approx& \frac{4M}{3}+ \frac{64M^2}{\pi^4}(1- \frac{4}{3}v_2)(v_{1,+}-v_{1,-})^2.    
\end{eqnarray}
最后 $\sigma^2(\Delta B_y)$ 和$\sigma^2(\Delta B_x)$之间的区别为:
\begin{eqnarray}
\sigma^2(\Delta B_y) - \sigma^2(\Delta B_x) &\approx& \frac{128M^2}{\pi^4}\Big[(1+\frac{\pi^2}{4M})\Delta\gamma_{112}-(\frac{4}{3}+\frac{\pi^2}{4M})v_2\Delta\delta\Big] \\
&\approx& \frac{128M^2}{\pi^4}(\Delta\gamma_{112}-\frac{4}{3}v_2\Delta\delta).
\end{eqnarray}

\setcounter{figure}{0}
\setcounter{table}{0}
\setcounter{equation}{0}

