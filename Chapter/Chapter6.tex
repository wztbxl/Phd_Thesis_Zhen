

\setcounter{section}{0}
%==========================================

\setcounter{figure}{0}
\setcounter{table}{0}
\setcounter{equation}{0}

\chapter{总结与展望}
% To add a non numbered chapter
%\addcontentsline{toc}{第一章}{相对论重离子碰撞}
% To insert this section on the table of contents

%\section{总结}
这篇论文总共可以分为三个部分。第一部分是关于电荷平衡函数法(Signed Balance Function)的模型检验,并对目前实验上用于寻找CME信号的三个基本的方法进行了系统的推导。第二部分对STAR实验组采集的Au+Au200 GeV 数据进行分析。第三部分对Isobar实验和STAR实验组是实施的盲分析法进行介绍,并利用盲分析中冻结的程序对模型数据进行灵敏度检测。

第一部分,首先对目前实验上寻找CME信号所用到的观测量的方法和现状进行来介绍。由于背景等贡献的影响,以目前的理论、实验研究结果并不能认为观测到来新CME已经被观测到来。近几年的主要工作就是研究如何更好的消除或者压低背景带来的影响。接下来对新提出的电荷平衡函数法进行了解读与检验,它主要是通过电荷平衡函数对碰撞中磁场所引起的横动量方向的波动来寻找CME。通过对Toy模型中不同信号、背景混合的情况下的研究方检验发现\rrest 和\rb 对CME信号的反应都很灵敏,并且两者对背景有着不同的反应:在\rrest 随着背景的增大而增大时\rb 会随着背景的增大而减小,通过这一特性该方法可以很好的对信号、背景进行区分。也就是说该方法的模型检测结果说明它可以用来寻找CME。最后对实验上目前用于寻找CME的最基本的三种方法进行了系统的推导,推导结果说明三种方法之间是相互关联的,它们核心部分的比较结果表明三种实验室的观测量对于相同的CME信号、背景的反应是彼此等同的。并没有哪种研究方法比另外的方法更要有优势的说法。


第二部分利用电荷平衡函数法对STAR 所采集的run16 Au + Au 200 GeV 实验数据进行分析。分析结果显示在所有的中心度中$r_{\mathrm{rest}}$ 、 $r_{\mathrm{lab}}$  和  $R_{\mathrm{B}}$的值都大于1,并且在$30-40\%$中心度中,实验的结果比EBE-AVFD中无信号的点要大。需要注意的是,SBF方法中$R_{\mathrm{B}}$在有背景的情况下是会被压低的。\rrest 和\rb 同时大于1的情况下很难认为这一结果是完全由背景的贡献而导致的。LCC对这一个新方法的影响还尚不明确,需要更深入的研究。


第三部分对STAR实验所进行的Isobar实验,并对STAR实验组所实施的对Isobar数据的盲分析法进行了介绍。
并利用EBE-AVFD模型所产生的Isobar的数据对这三种方法的灵敏度进行了系统的研究对比。需要指出的是为了排除分析者计算过程中导致的错误而引起不必要的偏差,在这一过程中对于$ \gamma$关联方法、R关联方法的分析程序直接利用STAR Isobar盲分析中所冻结的程序,电荷平衡函数法的分析流程则是与Quark Matter 2019所采用的对模型的一样的处理方法(SBF方法不参与盲分析)。EBE-AVFD模型Isobar的结果表明三种方法都对CME信号有一定的灵敏度。并且对于$Ru/Zr$显著性分析结果中发现$\Delta \gamma_{112}$与$r_{\mathrm{lab}}$有着及其相似的显著性,而与此同时R关联方法的$ \sigma_{R2}^{-1}$ 对不同的CME信号没有表现很强的区别。这一结果可以为分析STAR Isobar 实验结果提供重要的参考。